% Gemini theme
% https://github.com/anishathalye/gemini

\documentclass[final]{beamer}

% ====================
% Packages
% ====================

\usepackage{iftex}
\ifPDFTeX
  \usepackage[T1]{fontenc}
  \usepackage{lmodern}
\else
  \usepackage{fontspec}
  \usepackage{xeCJK}
%   \setmainfont{Helvetica Neue}
%   \setsansfont{Helvetica Neue}
%   \setmonofont{Helvetica Neue}
%   \setCJKmainfont{Songti SC}
  \setCJKmainfont[
  BoldFont=PingFang SC Semibold,
  ItalicFont=PingFang SC,
  BoldItalicFont=PingFang SC Semibold
]{PingFang SC}
  \setCJKsansfont{PingFang SC}
  \setCJKmonofont{PingFang SC}
\fi
\usepackage[size=custom,width=84.1,height=118.9,scale=1.0]{beamerposter}
\usetheme{gemini}
\usecolortheme{mit}
\usepackage{graphicx}
\usepackage{booktabs}
\usepackage{tikz}
\usepackage{pgfplots}
\pgfplotsset{compat=1.14}
\usepackage{anyfontsize}
\usepackage{listings}

% ====================
% Lengths
% ====================

% If you have N columns, choose \sepwidth and \colwidth such that
% (N+1)*\sepwidth + N*\colwidth = \paperwidth
\newlength{\sepwidth}
\newlength{\colwidth}
\setlength{\sepwidth}{0.025\paperwidth}
\setlength{\colwidth}{0.3\paperwidth}

\newcommand{\separatorcolumn}{\begin{column}{\sepwidth}\end{column}}

% ====================
% Title
% ====================

\title{二维(伪)拓扑图生成器}

\author{余政希 \inst{1}}

\institute[shortinst]{\inst{1}同济大学 \ 国豪书院}

% ====================
% Footer (optional)
% ====================

\footercontent{
  \href{https://github.com/yzxoi/topology-diagram-generator}{https://github.com/yzxoi/topology-diagram-generator} \hfill
  高级语言程序设计(进阶)课程作业 \hfill
  \href{mailto:2452633@tongji.edu.cn}{2452633@tongji.edu.cn}}
% (can be left out to remove footer)

% ====================
% Logo (optional)
% ====================

% use this to include logos on the left and/or right side of the header:
% \logoright{\includegraphics[height=7cm]{logo1.pdf}}
% \logoleft{\includegraphics[height=7cm]{logo2.pdf}}
\logoright{\includegraphics[height=7cm]{logos/Tongji_Uni_logo.svg.png}}

% ====================
% Body
% ====================

\begin{document}

\begin{frame}[t,fragile]
\begin{columns}[t]
\separatorcolumn

\begin{column}{\colwidth}

  \begin{block}{Overview}

    本项目实现了一个跨平台的二维(伪)拓扑图生成工具,
	支持通过参数化规则快速生成具有不同层次、不同连接、不同纹理特征的网状结构图案。
	本项目提供实时渲染与交互浏览(缩放、平移、节点拖拽),
	并支持一键导出为 PNG/SVG,用于算法可视化、图形设计草图与教学演示等。
	本项目基于 C++17 与 Qt6,并将“生成算法”和“可视化交互”打包为不同模块,
	便于扩展更多生成模式与样式预设。
	\end{block}

	\begin{figure}
    \centering
    \begin{tabular}{ccc}
      \includegraphics[width=0.32\linewidth]{figures/topology-1.png} &
      \includegraphics[width=0.32\linewidth]{figures/topology-2.png} &
      \includegraphics[width=0.32\linewidth]{figures/topology-3.png} \\
      \includegraphics[width=0.32\linewidth]{figures/topology-4.png} &
      \includegraphics[width=0.32\linewidth]{figures/topology-5.png} &
      \includegraphics[width=0.32\linewidth]{figures/topology-6.png} \\
      \includegraphics[width=0.32\linewidth]{figures/topology-Concentric-1.png} &
      \includegraphics[width=0.32\linewidth]{figures/topology-Concentric-2.png} &
      \includegraphics[width=0.32\linewidth]{figures/topology-Concentric-3.png} \\
    \end{tabular}
    \caption{\small  Layered Polygon Mode(第一、二排),Concentric Mode(第三排)}
  \end{figure}

	\begin{alertblock}{Highlights}
    \begin{itemize}
      \item \textbf{双生成范式:} 包含Concentric Mode(同心环网状)与 Layered Polygon Mode(分层多边形/星形结构)两种不同生成范式。
      \item \textbf{规则可组合:} 支持层内封圈、对角/星形、层间 1$\rightarrow$1 最近、1$\rightarrow$2 扩展、左右交替 Zigzag 等自定义,可通过参数自由组合形成丰富图案。
      \item \textbf{交互即数据:}节点可拖拽,边实时跟随;用于“生成后再编辑”的交互式探索流程。
      \item \textbf{所见即所得:} PNG 位图与 SVG 矢量一键导出。
      \item \textbf{模块易扩展:} Generator 仅负责 $V,E$,Scene/View 负责图元与交互,便于新增扩展模块。
    \end{itemize}
  \end{alertblock}
  \begin{block}{Quickstart}

\textbf{本项目依赖以下环境}

\begin{enumerate}
	\item Cmake >= 3.21
	\item C++17 编译器(clang++ / g++ / MSVC)
	\item Qtt6
  \end{enumerate}

  \vspace{0.3em}
  \textbf{macOS(Homebrew)}
  \begin{lstlisting}
brew install cmake qt@6
cmake -S . -B build -DCMAKE_BUILD_TYPE=Release
cmake --build build -j 8
./build/src/topology_diagram_generator
  \end{lstlisting}

  \textbf{Ubuntu / Debian}
  \begin{lstlisting}
sudo apt update
sudo apt install -y build-essential cmake \
     qt6-base-dev qt6-base-dev-tools
cmake -S . -B build -DCMAKE_BUILD_TYPE=Release
cmake --build build -j 8
./build/src/topology_diagram_generator
  \end{lstlisting}

  \textbf{Windows(VS 2022)}

  推荐使用 Visual Studio 2022,确保已安装并正确配置:
  \begin{enumerate}
	\item Visual Studio 2022(包含 “Desktop development with C++” 工作负载)
	\item Qt 6.x for MSVC 2022 64-bit
	\item Qt VS Tools
	\item CMake 3.21+,通常在安装 Visual Studio 2022 时包含。
  \end{enumerate}

打开 Visual Studio 2022 后,选择“打开文件夹(Open Folder)”并指定本项目根目录。

若 Qt VS Tools 插件配置正确,Visual Studio 将自动检测并加载项目的 CMake 配置文件。此时只需点击上方的“生成并运行”(Build and Run)按钮,即可直接编译并启动项目。

\vspace{0.9em}

本项目已在 Windows 11 24H2(x64) + Visual Studio 2022 + Qt 6.10.0 + Qt Visual Studio Tools、macOS(Qt 6.9.3 + Visual Studio Code + CMake 4.1.2) 以及 Ubuntu(Qt 6.6.3) 等环境下成功编译通过。

  \end{block}
\end{column}

\separatorcolumn

\begin{column}{\colwidth}

  \begin{block}{System Structure}

  系统采用四层架构设计,确保核心算法与界面显示完全解耦:

\begin{enumerate}
	\item \textbf{UI 控制层 (MainWindow)}:监听参数变化,组装 Params 结构体,驱动视图更新。
	\item \textbf{视图管理层 (TopologyView / Scene)}:作为“画布管理器”。View 负责视口交互(缩放/平移);Scene 负责图元的生命周期管理(创建、清空、索引)。
	\item \textbf{图元表现层 (NodeItem / EdgeItem)}:基于 Qt Graphics View 框架,实现具体的绘制逻辑与鼠标事件响应。
	\item \textbf{核心算法层 (TopologyGenerator)}:执行几何计算,输出无状态的 std::vector<Node> 和 Edge 数据,不依赖任何 UI 组件。
\end{enumerate}
   \begin{figure}
	\centering
	\includegraphics[width=0.99\linewidth]{figures/program-main.png}
	\caption{\small 程序运行主界面}
   \end{figure}
  \end{block}

  \begin{block}{Algorithm - Concentric Mode}

   用若干条同心“环”快速生成网状结构。每一环在圆周上进行等角采样,并对角度加入少量随机抖动以产生更自然的艺术效果;可选择是否将每一环封成多边形“环边”,以及是否将外层点连接到内层的两个最近点,从而形成稠密的网状连接。

\textbf{输入参数}
\begin{itemize}
  \item \textbf{Radii per Ring}:各环半径序列(从外到内或内到外均可,按输入顺序生成)。
  \item \textbf{Sides per Ring}:各环点数(即第 $i$ 环为“$n_i$ 边形”对应的 $n_i$)。
  \item \textbf{Ring Jitter}:每个点角度抖动(单位:度),用于增强随机性/艺术化。
  \item \textbf{Connect nodes in round}:是否绘制每一环的“环边”(首尾相连)。
  \item \textbf{Connect two nearest}:是否将每个外环点连接到内环的两个最近点。
\end{itemize}

\textbf{生成方式}
对第 $i$ 个环:
\begin{enumerate}
  \item 取半径 $r=r[i]$,点数 $n=n_{\text{sides}}[i]$。
  \item 第 $j$ 个点($0\le j<n$)角度为
  \[
    \theta_{i,j}=\frac{2\pi j}{n}+U(-\delta,+\delta),
  \]
  其中 $\delta$ 为由输入抖动角度(度)换算得到的弧度。
  \item 将极坐标转为直角坐标并压入节点表,同时把该环的节点 id 记录到 \texttt{rings[i]}。
\end{enumerate}
\emph{说明:}这里的“多边形”本质上是对圆周做等角采样形成的折线近似。

\textbf{复杂度分析}
\begin{itemize}
  \item 点生成:$\mathcal{O}\!\left(\sum_i n_i\right)$
  \item 层内环边:$\mathcal{O}\!\left(\sum_i n_i\right)$
  \item 层间连接(相邻两环):每对相邻环为 $\mathcal{O}(n_i\cdot n_{i+1})$
\end{itemize}

\vspace{0.6em}

  \end{block}

\begin{alertblock}{Challenges \& Solutions}
	
\begin{itemize}
\item \textbf{难点 A: 非均匀采样下的层间连接跳变} \\
\textit{解决方案:} 引入局部邻域搜索。

对每层节点计算 \texttt{atan2} 极角序列;层间匹配时,先通过角度寻找“主最近点”,再基于索引在其左右邻域进行 1-2 扩展,避免了全局搜索的不稳定性。

\vspace{0.5em} 

\item \textbf{难点 B: 高频参数调节导致的渲染卡顿} \\
\textit{解决方案:} 采用逻辑渲染分离 + 批量更新策略。

\texttt{Generator} 仅输出纯 C++ struct (Node/Edge);\texttt{Scene} 接收数据后,先 \texttt{clear()} 旧图元,再批量 \texttt{addItem()},最后仅触发一次 \texttt{update()},杜绝碎片化重绘。

\vspace{0.5em}

\item \textbf{难点 C: 导出图片与屏幕显示坐标不一致} \\
\textit{解决方案:} 统一坐标系。

使用 \texttt{mapToScene(viewport()->rect()).boundingRect()} 获取当前视口的精确场景坐标,将其作为 \texttt{QImage} 和 \texttt{QSvgGenerator} 的渲染源区域 (\texttt{sourceRect})。
    \end{itemize}

\end{alertblock}

  
\end{column}

\separatorcolumn

\begin{column}{\colwidth}

\begin{block}{Algorithm - Layered Polygon Mode}

    按“外$\rightarrow$内”绘制多层 $n_l$ 边形;每条边可细分 $0/1/2$ 个额外点(支持对称偏移)。
层内可选绘制环边与星形/对角线;层间连接支持最近 $1\!\rightarrow\!1$ 与扩展 $1\!\rightarrow\!2$
(并可按奇偶交替左右形成“之”字形)。中心点可设置为必连最内层。

\textbf{输入参数}
\begin{itemize}
  \item \textbf{Num Layers}:层数(建议 3--4)。
  \item \textbf{Layer N Sides}:第 $l$ 层多边形边数 $n_l\ge 3$。
  \item \textbf{Layer Radii}:第 $l$ 层半径 $R_l$。
  \item \textbf{Layer Phase Deg}:第 $l$ 层起始角(度),用于层间错位旋转(如五边形相差 $36^\circ$)。
  \item \textbf{Layer Subdiv Per Edge} $\in\{0,1,2\}$:每条边细分点数量。
  \item \textbf{Layer Subdiv Offset} $p\in[0,0.5]$:细分点插值比例($0.5$ 为中点;$p$ 与 $1-p$ 成对对称)。
  \item \textbf{Layer Draw Rim}:是否绘制本层环边(按顺序封圈)。
  \item \textbf{Layer Draw Diagonals / skipK}:是否绘制星形/对角线(跳步 $k$ 连接)。
  \item \textbf{Inter-Layer 1-1}:是否连接到内层最近的 1 个点。
  \item \textbf{Inter-Layer 1-2}:在 $1\!\rightarrow\!1$ 基础上再连接其相邻点之一(总计两条)。
  \item \textbf{Inter-Layer Zigzag}:若开启,两条邻边按外层点奇偶交替选择左/右邻点。
  \item \textbf{Jitter Deg / seed / Connect to Center}:角度微抖动、随机种子、以及中心点是否必连最内层。
\end{itemize}

\textbf{生成方式}
对第 $l$ 层:
\begin{enumerate}
  \item 生成正 $n_l$ 边形顶点(半径 $R=R_l$,带相位偏移)。
  \item 沿每条边 $\overline{AB}$ 追加细分点(线性插值):
  \[
    \text{subdiv}=1:\ (1-p)A+pB \quad (\text{或 } (A+B)/2),
  \]
  \[
    \text{subdiv}=2:\ (1-p)A+pB \ \text{与}\ pA+(1-p)B,
  \]
  二者关于边中点对称。
  \item 按边顺序将“端点 + 细分点”压入 \texttt{ringIds[l]},
        并保存每个点的极角
  \[
    \alpha=\operatorname{atan2}(y,x)
  \]
  到 \texttt{ringAng[l]} 作为层间匹配依据。
\end{enumerate}

\textbf{连边规则}
\begin{itemize}
  \item \textbf{层内环边}(\texttt{drawRim=true}):按 \texttt{ringIds[l]} 顺序相邻相连并封圈。
  \item \textbf{层内对角/星形}(\texttt{drawDiagonals=true}):以跳步 \texttt{skipK} 在同环内连接。
  \item \textbf{层间 $1\!\rightarrow\!1$ 最近连接}:外层点连接到内层角度最近点(用 $\alpha$ 近邻匹配)。
  \item \textbf{层间 $1\!\rightarrow\!2$ 扩展}:再连接该最近点的左/右邻居之一。
  \item \textbf{Zigzag 交替}:若开启,外层点按奇偶交替选择左/右邻居,形成交错网格效果。
  \item \textbf{中心连接}:中心点与最内层所有点相连(若 \texttt{Connect to Center} 启用)。
\end{itemize}

\textbf{复杂度分析}
\begin{itemize}
  \item 点生成:$\mathcal{O}\!\left(\sum_l n_l\cdot(1+\text{subdiv}_l)\right)$
  \item 层内环边/对角:$\mathcal{O}\!\left(\sum_l M_l\right)$,
        其中 $M_l$ 为第 $l$ 层实际点数。
  \item 层间连接(相邻两层):$\mathcal{O}\!\left(M_l\cdot M_{l+1}\right)$
\end{itemize}

  \end{block}


\begin{block}{Insights}
	在本项目中,我体会到:参数化生成可以让复杂结构被压缩为简洁的少量可控变量,而 Qt 的 Graphics View 框架则天然提供了稳定的渲染与交互基座。通过将节点与边的生成逻辑与渲染逻辑解耦,能够快速迭代新的连接规则与视觉风格,并保持兼容性与可扩展性。
	
	未来我希望继续引入更多生成范式(如力导向布局、约束优化),让本项目在可视化与设计探索中更进一步。

\end{block}

    \begin{block}{Acknowledgement}
	特别感谢 \textbf{Gemini} 与 \textbf{ChatGPT} 系列大语言模型。作为我的变成助手,它们在 Qt 调试、CMake 跨平台构建配置以及本海报的文案润色中提供了关键协助,极大地加速了从灵感到原型的迭代效率。
  \end{block}

\begin{exampleblock}{Who am I?}
\begin{minipage}[t]{0.36\textwidth}
  \vspace{0pt}
  \centering
  \includegraphics[width=0.95\linewidth]{figures/author.jpg}
\end{minipage}\hfill
\begin{minipage}[t]{0.60\textwidth}
  \vspace{0pt}
  我是余政希,同济大学国豪书院 2024 级人工智能(精英班)本科生,热爱计算机与人工智能方向。入学以来,我积极参与多项科研课题与工程项目实践,注重将理论学习与真实场景应用相结合,持续提升专业能力与创新素养。期待在技术创新与实践探索中不断突破自我,为团队协作与共同成长贡献力量。
\end{minipage}

\vspace{0.3em}

\textbf{Email:} \href{mailto:2452633@tongji.edu.cn}{2452633@tongji.edu.cn}

\textbf{Github:} \href{https://github.com/yzxoi}{Github/yzxoi}
\end{exampleblock}

\end{column}

\separatorcolumn
\end{columns}
\end{frame}

\end{document}
